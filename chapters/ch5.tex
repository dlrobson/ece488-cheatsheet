\section*{5.1 Basic State-Space Framework and Manipulations}
State-space models are not unique. Infinite number of state-space realizations that correspond to the same transfer functions
They follow the form:
\begin{gather*}
    \begin{aligned}
        \dot{x} & = Ax + Bu \\
        y       & = Cx + Du
    \end{aligned}
\end{gather*}
Manipulation 1: Convert from SS to TF
\[
    Y(s) = [ D + C ( sI - A )^{-1} B ] \cdot U(s)
\]
Manipulation 2: Poles and zeros of a SS system
\begin{itemize}
    \item Poles of the SS system are \( eig(A) == \) roots of \( det( sI - A ) \)
    \item Zeroes are the values of s where the rank of the following matrix falls below its normal rank:
          \[
              R(s) = \begin{bmatrix}
                  (sI - A) & -B \\
                  C        & D
              \end{bmatrix}
          \]
          Which can be computed by finding the the roots of \( det( R(s) ) \)
\end{itemize}
Manipulation 3: Interconnecting state-space models
\fig[0.8]{manipulation3_question}
\fig[0.8]{manipulation3_solution}
Manipulation 4: Converting from TF to SS
\begin{itemize}
    \item SISO case, creates a SS with controllable canonical form (CCF):
\end{itemize}
\begin{equation*}
    \begin{aligned}
        \dot{x} & = \begin{bmatrix}
            0      & 1      & 0      & \cdots & 0        \\
            0      & 0      & 1      &        & 0        \\
            0      & 0      & 0      & \ddots & \vdots   \\
            \vdots & \vdots & \vdots &        & 1        \\
            -a_0   & -a_1   & -a_2   & \cdots & -a_{n-1}
        \end{bmatrix} x +
        \begin{bmatrix}
            0      \\
            0      \\
            \vdots \\
            0      \\
            1
        \end{bmatrix} u                                        \\
        y       & =\begin{bmatrix}
            c_0 & c_1 & c_2 & \cdots & c_{n-1}
        \end{bmatrix}x + \begin{bmatrix}
            d
        \end{bmatrix}u
    \end{aligned}
\end{equation*}

\begin{itemize}
    \item SIMO case:
          \[
              G(s) = \begin{bmatrix}
                  A_1    & 0   & \cdots & B_1    \\
                  0      & A_2 &        & B_2    \\
                  \vdots &     & \ddots & \vdots \\
                  C_1    & 0   & \cdots & D_1    \\
                  0      & C_2 &        & D_2    \\
                  \vdots &     & \ddots & \vdots \\
              \end{bmatrix}
          \]
    \item MIMO case:
          \[
              G(s) = \begin{bmatrix}
                  A_1    & 0   & \cdots & B_1    & 0   & \cdots \\
                  0      & A_2 &        & 0      & B_2 &        \\
                  \vdots &     & \ddots & \vdots &     & \ddots \\
                  C_1    & C_2 & \cdots & D_1    & D_2 & \cdots \\
              \end{bmatrix}
          \]
\end{itemize}
Manipulation 5: Block Diagram of SS Model

\fig[0.8]{manipulation5}

\section*{5.2 Controllability, Observability, Stabilizability, and Detectability}
Modes that show up on the output y(t) are said to be \textit{Observable}.
If all modes are observable, the state-space realization is said to \textit{observable}.
If not, it's \textit{unobservable}

Modes that are affected by the input are said to \textit{controllable}
If all modes are controllable, the state-space realization is said to \textit{controllable}.
If not, it's \textit{uncontrollable}

A state-space realization where \underline{all unstable} modes are controllable is said to be \textit{stabilizable}.
\begin{equation*}
    \text{controllability} \Rightarrow \text{stabilizability}
\end{equation*}

A state-space realization where \underline{all unstable} modes are observable is said to be \textit{detectable}.
\begin{equation*}
    \text{observability} \Rightarrow \text{detectability}
\end{equation*}

A system is controllable and observable iff the corresponding TF matrix has no P-Z cancellations

A system is stabilizable and detectable iff the corresponding TF matrix has no \underline{unstable} P-Z cancellations

\section*{5.2.2 Controllability and Stabilizability}
\( A, B \) is controllable iff the n x (nm) \textit{controllability matrix} has rank \( n \).
\begin{equation*}
    M_c = \begin{bmatrix}
        B & AB & A^2B & \cdots & A^{n-1}B \\
    \end{bmatrix}
\end{equation*}

Using the PBH controllability test, \( A, B \) is controllable iff
\begin{equation*}
    \text{rank}\begin{bmatrix}
        A - \lambda I & B \\
    \end{bmatrix} = n \text{ for all } \lambda \in \text{eig}(A)
\end{equation*}
Recall: eig(A) is found as the solutions to \( \lambda \), for det\( ( A - \lambda I ) \)

\section*{5.2.3 Observability and Detectability}
\( A, C \) is observable iff the (pn) x n \textit{observability matrix} has rank \( n \).
\begin{equation*}
    M_o = \begin{bmatrix}
        C        \\
        CA       \\
        CA^2     \\
        \vdots   \\
        CA^{n-1} \\
    \end{bmatrix}
\end{equation*}

Using the PBH observability test, \( A, C \) is controllable iff
\begin{equation*}
    \text{rank}\begin{bmatrix}
        A - \lambda I \\
        C             \\
    \end{bmatrix} = n \text{ for all } \lambda \in \text{eig}(A)
\end{equation*}
Recall: eig(A) is found as the solutions to \( \lambda \), for det\( ( A - \lambda I ) \)

\section*{5.3 Plain State Feedback Control (Pole Placement)}
The plain state-feedback controller \( u = -Kx + r \) can arbitrarily place poles.
The plant \textbf{must be controllable}.
If it is uncontrollable but stabilizable, we cannot place poles arbitrarily but we can find a \( K \) to stabilizie \( A - BK \).

K is a row vector.

Steps:
\begin{equation*}
    \begin{aligned}
        A_{CLS}      & = A - BK          \\
        \pi(s)       & = det(sI-A_{CLS}) \\
        \pi_{des}(s) & = \pi(s)          \\
    \end{aligned}
\end{equation*}

CLS poles are at eig(\( A_{CLS} \)) = eig(\( A - BK \))

\section*{5.4 Observer}
Consider the following Plant:
\begin{equation*}
    \begin{aligned}
        \dot{x} & = Ax + Bu \\
        y       & = Cx + Du \\
    \end{aligned}
\end{equation*}
and Observer
\begin{equation*}
    \begin{aligned}
        \dot{\hat{x}} & = A\hat{x} + Bu + H(y - \hat{y}) \\
        \hat{y}       & = C\hat{x} + Du                  \\
    \end{aligned}
\end{equation*}
The estimation error is given by \( e(t) = x(t) - \hat{x}(t) \). Then:
\begin{itemize}
    \item \( \dot{e} = (A - HC)e \), and \( e(t = \infty) = 0 \) for any IC \( e(0) \) iff eig(\( A - HC \)) \( \in \) OLHP.
    \item The poles of eig(\( A - HC \)) can be placed arbitrarily iff (A, C) is \underline{observable}.
    \item The poles of eig(\( A - HC \)) can be placed in OLHP iff (A, C) is only \underline{detectable}.
\end{itemize}

H is a column vector.

Steps:
\begin{equation*}
    \begin{aligned}
        \pi(s)       & = det(sI - (A - HC)) \\
        \pi_{des}(s) & = \pi(s)             \\
    \end{aligned}
\end{equation*}


Resulting state-space realization:
\begin{equation*}
    \begin{aligned}
        \begin{bmatrix}
            \dot{x}       \\
            \dot{\hat{x}} \\
        \end{bmatrix}
                & = \begin{bmatrix}
            A  & 0    \\
            HC & A-HC \\
        \end{bmatrix}\begin{bmatrix}
            x       \\
            \hat{x} \\
        \end{bmatrix} + \begin{bmatrix}
            B \\
            B
        \end{bmatrix} u \\
        \bar{y} & = I \cdot \begin{bmatrix}
            x       \\
            \hat{x} \\
        \end{bmatrix}
    \end{aligned}
\end{equation*}

\section*{5.4 Observer-based state-feedback control}
Mixture of solving for K and H, using the above steps. The CLS for the observer-based state-feedback controller:
\begin{equation*}
    \begin{aligned}
        \begin{bmatrix}
            \dot{x} \\
            \dot{e} \\
        \end{bmatrix}
          & = \begin{bmatrix}
            A-BK & BK   \\
            0    & A-HC \\
        \end{bmatrix}\begin{bmatrix}
            x \\
            e \\
        \end{bmatrix} + \begin{bmatrix}
            B \\
            0
        \end{bmatrix} r \\
        y & = \begin{bmatrix}
            C & 0 \\
        \end{bmatrix} \begin{bmatrix}
            x \\
            e \\
        \end{bmatrix}
    \end{aligned}
\end{equation*}

The closed loop poles are found at eig(\(A-HC\)) \(\cup\) eig(\(A-BK\))

General rule-of-thumb is to place eig(\(A-HC\)) about twice as fast (i.e., real parts twice as large) as eig(\(A-BK\)).

K: Arbitrary pole placement is possible iff (A, B) is controllable, and stabilization is possible iff (A, B) is stabilizable.

H: Arbitrary placement of poles of the observer error-dynamics is possible iff (A, C) is observable, and they can be placed (not necessarily arbitrarily) in the OLHP iff (A, C) is detectable.

\section*{5.5 Plain state-feedback controller with integral action}

\fig[1.0]{5.5-integral-control}
\begin{equation*}
    \begin{aligned}
        \begin{bmatrix}
            \dot{x} \\
            \dot{z} \\
        \end{bmatrix} & = \begin{bmatrix}
            A & 0 \\
            C & 0 \\
        \end{bmatrix}\begin{bmatrix}
            x \\
            z \\
        \end{bmatrix} + \begin{bmatrix}
            B \\
            0 \\
        \end{bmatrix}u + \begin{bmatrix}
            0  \\
            -I \\
        \end{bmatrix}r \\
        \dot{\tilde{x}}            & = \tilde{A}\tilde{x} + \tilde{B}u + \tilde{B}_r r                                                                  \\
    \end{aligned}
\end{equation*}

The CLS poles (eig(\( \tilde{A} - \tilde{B} K \))) can be placed arbitrarily iff (\( \tilde{A}, \tilde{B}\)) is controllable
Essentially, the same steps as in 5.3, however \( A \) is replaced with \( \tilde{A} \).
Still need to calculate the characteristic polynomial using det(\( sI - (\tilde{A} - \tilde{B}K) \))

\section*{5.5 Observer-based state feedback controller with integral action}

Essentially, the same steps as in 5.4, however \( A \) is replaced with \( \tilde{A} \).

CLS poles: eig(\(\tilde{A}-\tilde{B}K\)) \(\cup\) eig(\(A - HC\))

K: Arbitrary pole placement is possible iff (\(\tilde{A}, \tilde{B}\)) is controllable.
Stabililization is possible iff \(\tilde{A}, \tilde{B}\) is stabilizable.

H: Arbitrary placement of poles of the observer error-dynamics is possible iff (A, C) is observable.
They can be placed (not necessarily arbitrarily) in the OLHP iff (A, C) is detectable.

\section*{5.6 Optimal State-feedback control}

Where \( Q \geq 0 \) and \( R > 0 \), minimize:
\begin{equation*}
    J=\int_0^\infty x(t)^T Q x(t) + u(t)^T R u(t) dt
\end{equation*}
Assume that (A, B) is stabilizable and (A, Q) is detectable.

The resulting controller is the stabilizing linear state-feedback controller, \( u(t) = -Kx(t) \), where K is computed by finding the unique symmetric PSD solution P to the algebraic Riccati equation (ARE):
\begin{equation*}
    A^T P + PA - PBR^{-1}B^TP + Q = 0
\end{equation*}
and setting \( K = R^{-1}B^T P\).

Q and R are tuning knobs. Q is commonly a diagonal matrix where \( q_{ii} \geq 0 \).
This results in:
\begin{equation*}
    x(t)^T Qx(t) = q_{11}x_1^2(t) + q_{22}x_2^2(t) + \cdots + q_{nn} x_n^2(t)
\end{equation*}
Q is commonly chosen as \( Q = C^T C \). This leads to:
\begin{equation*}
    \begin{aligned}
        y(t)         & = Cx(t)      \\
        x(t)^T Qx(t) & = ||y(t)||^2
    \end{aligned}
\end{equation*}

R is also commonly a diagonal matrix.

Decreasing R minimizes the control effort, and the poles go off to \( -\infty \).
Increasing R increases the control effort, and the poles approach the OL poles.

Changing R does the opposite effect.

Increasing Q and R by the same factor does not change K or the pole locations.

\section*{2.1 Laplace Transforms}
\begin{equation*}
    \begin{aligned}
        u_{step}(t)        & \Rightarrow \frac{1}{s}         \\
        u_{ramp}(t)        & \Rightarrow \frac{1}{s^2}       \\
        e^{at}u_{step}(t)  & \Rightarrow \frac{1}{s - a}     \\
        sin(at)u_{step}(t) & \Rightarrow \frac{a}{s^2 + a^2} \\
        cos(at)u_{step}(t) & \Rightarrow \frac{s}{s^2 + a^2} \\
        \delta(t)          & \Rightarrow 1                   \\
    \end{aligned}
\end{equation*}

\section*{2.1 Bode Plots}

\fig[0.825]{2.1-bode}

\fig[0.825]{2.1-bode2}

\section*{2.2 Routh-Hurwitz}

BIBO stable iff proper and all poles lie in OLHP.

Given:
\begin{equation*}
    Q(s) = a_n s^n + a_{n-1}s^{n-1} + \cdots + a_1 s + a_0
\end{equation*}

\fig[0.9]{2.2-routh-hurwitz}
\begin{equation*}
    b_1 = \frac{a_{n-1}a_{n-2} - a_n a_{n-3}}{a_{n-1}}
\end{equation*}

All roots of Q(s) are in the OLHP iff all elements in the first column do not have any sign changes.

\section*{2.3 Steady-state Calculations}

FVT states:
\begin{equation*}
    f_\infty = \lim_{s \rightarrow 0}sF(s)
\end{equation*}

For a standard controller plant combination, the following equations can apply
\begin{equation*}
    \begin{aligned}
        y_\infty = T_{ry}(0) = \lim_{s \rightarrow 0} \frac{PC}{1 + PC}
        e_\infty = T_{re}(0) = \lim_{s \rightarrow 0} \frac{1}{1 + PC}
    \end{aligned}
\end{equation*}

\section*{2.4 Transient Response Calculations}

Dominant Dynamics: the closer the poles and zeros are to the origin, the more important they are.

Poles and zeros close together in the OLHP essentially cancel each other out

First order systems equations:
\begin{equation*}
    H(s) = \frac{1}{\tau s + 1}
\end{equation*}
The 2\% settling time is about \( T_s \approx 4 \tau \).

Second-order systems:
\begin{equation*}
    H(s) = \frac{\omega_n^2}{s^2 + 2 \zeta \omega_n s + \omega_n^2}
\end{equation*}
System is underdamped when \( 0 < \zeta < 1 \).

\fig[0.8]{2.4-second-order}

\begin{tabular}{l r}
    2\% settling time     & \(T_s \approx \frac{4}{\zeta\omega_n}\)               \\
    Oscillation frequency & \( \omega_n \sqrt{1-\zeta^2}\)                        \\
    Overshoot             & \( OS = 100\exp{\frac{-\zeta\pi}{\sqrt{1-\zeta^2}}}\) \\
    Peak time             & \( T_p = \frac{\pi}{\omega_n\sqrt{1-\zeta^2}}\)       \\
\end{tabular}

OLHP poles make the reponse more sluggish.
OLHP zeros make the response more perky

Bandwidth relates to the speed of response.
Higher bandwidth means a faster speed of response.
Bandwidth is often defined to be the frequency where \( G(j\omega) \) is 3 dB less than \( | G(0) |\)
Bandwidth is also defined as the gain-crossover frequency, which is the frequency where \( |P(j\omega)C(j\omega)| = 0 dB \)

\section*{2.5 Root Locus Methods}

\begin{equation*}
    Q(s) = D(s) + K \cdot N(s)
    P(s)C(s) = \frac{N(s)}{D(s)}
\end{equation*}
m is the order of N(s)
n is the order of D(s)

Rules:
\begin{itemize}
    \item RL is symmetric about the real axis.
    \item RL plot consists of n branches.
    \item RL is a continuous function of K.
    \item The root locus starts at the roots of D(s)
    \item m branches terminate at the m roots of N(s). The remaining n-m branches go off to infinity. It is centered around \( \sigma \).
          \begin{equation*}
              \sigma = \frac{\Sigma(\text{roots of D(s)}) - \Sigma(\text{roots of N(s)})}{n-m}
          \end{equation*}
    \item no-yes-no rule. A test point s located on the real axis is part of the root locus if
          and only if the total number of (finite) real poles and zeros of P(s)C(s) to the right
          of s is odd.
\end{itemize}

\section*{2.6 Nyquist Plots}

If the number of CCW encirclements of the critical point \( s = -1 \) by the Nyquist plot of L(s) equals the number of OLS poles in the ORHP.

\section*{2.6 Phase Margin and Gain Margin}

\fig[0.95]{2.6-PM-GM}

\begin{equation*}
    PM \approx 100\zeta \text{ [deg]}
\end{equation*}

\section*{2.7 Lead, Lag, PID}

Lead compensation increases PM.

Lag compensation boosts low-frequency gain, or can increase PM.

PID control is simple controller and is commonly used.
\begin{equation*}
    \begin{aligned}
        C(s) & = K_P + \frac{K_I}{s} + \frac{K_D s}{\tau s + 1} \\
        C(s) & = \frac{K_D s^2 + K_P s + K_I}{s (\tau s + 1)}   \\
    \end{aligned}
\end{equation*}

\section*{2.8 Small Gain Theorem}

\fig[0.6]{2.8-SGT}

\(M(s)\) is known and stable

\(\Delta(s)\) is uncertain but stable with \( |\Delta(j\omega)| < 1 \) for all \( \omega \)
\begin{equation*}
    \left| \Delta(j\omega)M(j\omega) < 1 \right| \text{ for all } \omega \text{ and all admissible } \Delta(s)
\end{equation*}
or, equivalently,
\begin{equation*}
    \left| M(j\omega) \leq 1 \right| \text{ for all } \omega
\end{equation*}

\fig[0.9]{2.8-SGT-example}

\section*{2.9 Loopshaping}

\begin{equation*}
    L(s) = P(s)C(s)
\end{equation*}

At low frequencies, we want:
\begin{equation*}
    \left| L(j\omega) \right| \gg 1
\end{equation*}

At high frequencies, we want:
\begin{equation*}
    \left| L(j\omega) \right| \ll 1
\end{equation*}


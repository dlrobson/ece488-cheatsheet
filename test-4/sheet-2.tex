\documentclass[12pt]{article}
\usepackage[margin=0.25in]{geometry}
\usepackage{multicol, graphicx, mathtools}

\begin{document}
\begin{multicols*}{2}
    \underline{\smash{4.1 Basic MIMO Concepts}}
    \begin{gather*}
        \begin{aligned}
            y      & = ( I + PCH )^{-1} PCr \\
            L_o    & = PC                   \\
            S_o    & = (I + L_o) ^ {-1}     \\
            T_o    & = L_o(I + L_o) ^ {-1}  \\
            S_o(s) & + T_o(s) = I
        \end{aligned}
    \end{gather*}
    The \textit{minor} of a matrix is the determinant of any square submatrix obtained by deleting certain row's and/or columns of the matrix

    The \textit{rank} of a matrix is the max number of linearly independent column/row vectors.

    To find \(\phi(s)\), find all the minors. 1st order is just the element, while second order is as described above. Find the lowest common denominator of all non-zero minors.

    A MIMO system is stable iff all the individual elements of the system are stable, and if all MIMO system poles are in the OLHP

    The zero polynomial \(z(s)\) is the greatest common divisor of all the r-th order minors. Steps:
    \begin{enumerate}
        \item Find the minors for the MIMO system
        \item Determine \(\phi(s)\)
        \item Make each minor as a fraction over \(\phi(s)\)
        \item Find the GCD of the r-th order minors.
    \end{enumerate}

\end{multicols*}
\end{document}
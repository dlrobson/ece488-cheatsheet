\documentclass[12pt]{article}
\usepackage[margin=0.25in]{geometry}
\usepackage{multicol, graphicx, mathtools}

\begin{document}
\begin{multicols*}{2}
    \underline{\smash{3.1 Youla Parameterization}} \\
    For a stable plant. The set of all stabilizing controllers equals the following set:
    \[
        \left\{
        C(s) : C(s) = \frac{ Q(s) }{ 1 - P(s) Q(s) }
        \right\}
    \]
    where \( Q(s) \) is \underline{\smash{proper}} and \underline{\smash{stable}} \\

    \underline{\smash{Coprime Factorization}} \\
    Let \( P(s) \) be any proper TF. Then there exists \underline{\smash{stable}}
    \underline{\smash{proper}} TFs \( N_P(s) \), \( X_P(s) \), \( M_P(s) \), and \( Y_P(s) \)
    such that the following relationships hold for all s:
    \begin{gather*}
        P(s) = \frac{ N_P(s) }{ M_P(s) }\\
        N_P(s) X_P(s) + M_P(s) Y_P(s) = 1
    \end{gather*}

    \underline{\smash{Youla Parameterization for General \( P(s) \)}} \\
    For an unstable plant. Assume there are no CRHP pole-zero cancellations in \( P(s) \).
    Perform a coprime factorization on \( P(s) \).
    Then the set of all stabilizing controllers equals the following set:
    \[
        \left\{
        C(s) : C(s) = \frac{ X_P(s) + M_P(s)Q(s) }{ Y_P(s) - N_P(s) Q(s) }
        \right\}
    \]
    where \( Q(s) \) is \underline{\smash{proper}} and \underline{\smash{stable}} \\

    \underline{\smash{3.2 Performance Limitations}} \\
    At frequencies where \( | S(j\omega) | \ll 1 \),
    good tracking, good disturbance rejection, and good sensitivity. Exact opposite when
    \( | S(j\omega) | \gg 1 \).
    \begin{gather*}
        S(s) = \frac{ 1 }{ 1 + L(s) }
    \end{gather*}

    \underline{Ideally}:
    \begin{itemize}
        \item \( |S(j\omega)| \) is small when \( |L(j\omega)| \) is large
        \item \( |S(j\omega)| \) is near one (0 dB) when \( |L(j\omega)| \) is small
    \end{itemize}

    The complementary sensitivity, \( T(s) \), tells us that robust stability is achieved iff there is nominal stability and:
    \[
        \begin{aligned}
                            & | W(j\omega) T_0(j\omega) | < 1 \text{ for all } \omega               \\
            \Leftrightarrow & | T_0(j\omega) | < \frac{ 1 }{ |W(j\omega)| } \text{ for all } \omega
        \end{aligned}
    \]
    It also tells us how the closed-loop system responds to sensor noise.
    \[T(s) = \frac{ L(s) }{ 1 + L(s) } \]

    \underline{Ideally}:
    \begin{itemize}
        \item \( |T(j\omega)| \) is small (in fact, \( |T(j\omega)| \approx |L(j\omega)| \)) at high frequencies when sensor noise is significant, and little feedback effort is used
        \item \( |S(j\omega)| \) is near one (0 dB) at low frequencies, where lots of feedback effort is used
    \end{itemize}

    The relationship between the two equations:
    \begin{gather}
        S(j\omega) + T(j\omega) = \frac{ 1 }{ 1 + L(j\omega) } + \frac{ L(j\omega) }{ 1 + L(j\omega) } = 1 \\
        | S(j\omega) | + | T(j\omega) | \geq 1 \text{ for each } \omega
    \end{gather}
    Notes:
    \begin{itemize}
        \item tradeoff between the reasons for wanting \( | S(j\omega) | \) to be small and the reason for wanting \( | T(j\omega) | \)
        \item both cannot both be "good" (i.e., very small) at the same frequency, but it is possible for both to be "bad" (i.e., very large) at the same frequency
        \item We never want the magnitude bode plots of S(s) or T(s) to have large peaks
    \end{itemize}

    \underline{\smash{3.3 Interpolation Constraints}}
    Assume that the closed-loop system is stable.
    \begin{itemize}
        \item If \( P(s)C(s) \) has a CRHP zero at \( s = z \), then \( S(z) = 1 \) and \( T(z) = 0 \)
        \item If \( P(s)C(s) \) has a CRHP pole at \( s = p \), then \( S(p) = 0 \) and \( T(p) = 1 \)
    \end{itemize}

    \underline{\smash{3.4 Performance Limitations Due to ORHP}} \\
    Bandwidth rule of thumb is
    \[ \text{bandwidth } > 2p \]

    \underline{\smash{Lemma}} \\
    If:
    \begin{itemize}
        \item The closed-loop system is stable
        \item The plan has an open-loop unstable real pole at \( s = p > 0 \)
        \item The reference signal is a unit step
    \end{itemize}
    Then the tracking error, \( e(t) = r(t) - y(t) \) necessarily satisfies:
    \[
        \int_{0}^{\infty} e(t)e^{-pt} \,dt = 0
    \]
    An ORHP pole in the plant leads to overshoot in the closed-loop step response \\
    \begin{gather*}
        y_{OS} \geq (1 - 0.9y_\infty)(e^{pt_r} - 1) > 0 \\
        t_r \leq \frac{1}{p} ln \left( 1 + \frac{y_{OS}}{1 - 0.9y_\infty} \right)
    \end{gather*}

    \underline{\smash{Bode Sensitivy Integral}} \\
    Assume that:
    \begin{itemize}
        \item The controller stabilizes the closed-loop system
        \item The loop gain, \( L(s) = P(s) C(s) \), has a \textbf{\textit{relative degree}} of at least two (i.e., it has at least two more poles than zeros).
    \end{itemize}
    Let \( N_p \geq 0 \) denote the number of ORHP poles of \( L(s) \) and denote the ORHP poles by \( p_1, \ldots , p_{N_p} \). Then the sensitivity function must satisify
    \[
        \int_{0}^{\infty} ln|S(j\omega)| \,d\omega = \sum_{i = 1}^{N_p} Re(p_i) \geq 0
    \]
    If there are no unstable poles, then the equation reduces to
    \[
        \int_{0}^{\infty} ln|S(j\omega)| \,d\omega = 0
    \]
    Where the negative area equals the positive area through a \textit{waterbed effect}.

    If there are an unstable poles, then the area of sensitivity increase must be \textit{greater than} the area of sensitivity decrease.

    \underline{\smash{3.5 Performance Limitations Due to ORHP Zeroes}} \\
    If the plant has an ORHP real zero at s = z, then good performance can be achieved if the closed-loop bandwidth is significantly smaller than z. ORHP Zeroes:
    \begin{itemize}
        \item Zeroes near the origin are worse than ORHP zeroes far from the origin
        \item ORHP zeroes lead to \textit{undershoot} in the step responses
        \item You cannot get rid of CRHP zeroes, since \(P(s)C(s)\) is the numerator of \(T_{ry}(s)\)
    \end{itemize}
    \underline{\smash{Lemma}}
    \begin{gather*}
        \int_{0}^{\infty} y(t)e^{-zt} \,dt = 0 \\
        y_{US} \geq \frac{0.98y_\infty}{e^{zt_s} - 1} > 0 \\
        t_r \leq \frac{1}{z} ln \left( 1 + \frac{0.98y_\infty}{y_{US}} \right)
    \end{gather*}
    Bandwidth rule of thumb given a ORHP zero:
    \[ \text{bandwidth } < z/2 \]

    \underline{\smash{Poisson Integral}}
    \begin{gather*}
        \int_{0}^{\infty} ln|S(j\omega)|W(\omega) \,d\omega = \pi\sum_{i = 1}^{N_p} ln\left| \frac{p_i + z}{p_i^* - z} \right| \geq 0 \\
        W(\omega) = \frac{2 z}{ z^2 + \omega ^ 2}
    \end{gather*}

    % \underline{\smash{Additional Important Equations}}
    % \begin{gather*}
    %     \begin{aligned}
    %         \pi(s) & = N_P N_C + D_P D_C \\
    %         L(s)   & = P(s) C(s)         \\
    %     \end{aligned}
    % \end{gather*}
    % Given a unit step:
    % \begin{gather*}
    %     \begin{aligned}
    %         y_\infty & = T_{ry}(0) = \lim_{s \to 0} \frac{P(0)C(0)}{1 + P(0)C(0)} \\
    %         e_\infty & = T_{re}(0) = \lim_{s \to 0} \frac{1}{1 + P(0)C(0)}        \\
    %                  & = 1 - P(0)Q(0)
    %     \end{aligned}
    % \end{gather*}

    \underline{\smash{4.1 Basic MIMO Concepts}}
    \begin{gather*}
        \begin{aligned}
            y      & = ( I + PCH )^{-1} PCr \\
            L_o    & = PC                   \\
            S_o    & = (I + L_o) ^ {-1}     \\
            T_o    & = L_o(I + L_o) ^ {-1}  \\
            S_o(s) & + T_o(s) = I
        \end{aligned}
    \end{gather*}
    The \textit{minor} of a matrix is the determinant of any square submatrix obtained by deleting certain row's and/or columns of the matrix

    The \textit{rank} of a matrix is the max number of linearly independent column/row vectors.

    To find \(\phi(s)\), find all the minors. 1st order is just the element, while second order is as described above. Find the lowest common denominator of all non-zero minors.

    A MIMO system is stable iff all the individual elements of the system are stable, and if all MIMO system poles are in the OLHP

    The zero polynomial \(z(s)\) is the greatest common divisor of all the r-th order minors. Steps:
    \begin{enumerate}
        \item Find the minors for the MIMO system
        \item Determine \(\phi(s)\)
        \item Make each minor as a fraction over \(\phi(s)\)
        \item Find the GCD of the r-th order minors.
    \end{enumerate}

\end{multicols*}
\end{document}
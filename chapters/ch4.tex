\section*{4.1 Basic MIMO Concepts}
\begin{equation*}
    \begin{aligned}
        y      & = ( I + PCH )^{-1} PCr \\
        L_o    & = PC                   \\
        S_o    & = (I + L_o) ^ {-1}     \\
        T_o    & = L_o(I + L_o) ^ {-1}  \\
        S_o(s) & + T_o(s) = I
    \end{aligned}
\end{equation*}

The \textit{minor} of a matrix is the determinant of any square submatrix obtained by deleting certain row's and/or columns of the matrix

The \textit{rank} of a matrix is the max number of linearly independent column/row vectors.

To find \(\phi(s)\), find all the minors. 1st order is just the element, while second order is as described above. Find the lowest common denominator of all non-zero minors.

A MIMO system is stable iff all the individual elements of the system are stable, and if all MIMO system poles are in the OLHP

The zero polynomial \(z(s)\) is the greatest common divisor of all the r-th order minors. Steps:
\begin{enumerate}
    \item Find the minors for the MIMO system
    \item Determine \(\phi(s)\)
    \item Make each minor as a fraction over \(\phi(s)\)
    \item Find the GCD of the r-th order minors.
\end{enumerate}

\section*{4.2 Decentralized Control}
\begin{itemize}
    \item Assume plant is square
    \item Ignore cross-channel interactions in the plant, resulting in a MIMO controller that is diagonal
    \item If there is too much coupling, then the resulting system is unstable
\end{itemize}
\section*{4.3 Decoupling Control}
\begin{itemize}
    \item A decoupling block is added: \(P_{aug} = PW\)
    \item Design a decentralized controller for the augmented plant: \(C=W\bar{C}\)
\end{itemize}
Steps:
\begin{enumerate}
    \item Decouple the plant at dc. i.e. Solve \(P(0)\)
    \item Solve \(W(s)=P_1^{-1}(0)\), to get \( P_{aug} = P * W \)
    \item Solve for the controllers, ignoring the cross-channel interactions
    \item \(C(s) = W\bar{C}(s)\)
\end{enumerate}
\section*{4.4 Sequential Loop Closure}
\begin{itemize}
    \item Accounts for cross-channel interactions
    \item Feedback loops are closed sequentially instead of in parallel
    \item \textbf{This approach only applies if the plant has a triangular structure.}
    \item Ex:
          \begin{itemize}
              \item Given Plant:
                    \[
                        P(s) = \begin{bmatrix}
                            P_{11} & P_{12} \\
                            0      & P_{22}
                        \end{bmatrix}
                    \]
                    \(y_1 = P_{11}u_1 + P_{12}u_2\) and \(y_2 = P_{22}u_2\)
              \item first design \( C_{22} \) to stabilize \(P_22\) since it is independent of \( P_{11} \) and \( P_{12} \)
              \item Then design \( C_{22} \)
          \end{itemize}
\end{itemize}